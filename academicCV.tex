%%%%%%%%%%%%%%%%%%%%%%%%%%%%%%%%%%%%%%%%%
% Medium Length Professional CV
% LaTeX Template
% Version 2.0 (8/5/13)
%
% This template has been downloaded from:
% http://www.LaTeXTemplates.com
%
% Original author:
% Trey Hunner (http://www.treyhunner.com/)
%
% Important note:
% This template requires the resume.cls file to be in the same directory as the
% .tex file. The resume.cls file provides the resume style used for structuring the
% document.
%
%%%%%%%%%%%%%%%%%%%%%%%%%%%%%%%%%%%%%%%%%

%----------------------------------------------------------------------------------------
%	PACKAGES AND OTHER DOCUMENT CONFIGURATIONS
%----------------------------------------------------------------------------------------

\documentclass{resume} % Use the custom resume.cls style
\usepackage{bibentry}
\usepackage[left=0.4 in,top=0.4in,right=0.4 in,bottom=0.4in]{geometry} % Document margins
\newcommand{\tab}[1]{\hspace{.2667\textwidth}\rlap{#1}} 
\newcommand{\itab}[1]{\hspace{0em}\rlap{#1}}
\name{Logan Reed} % Your name
\address{(512) 839 - 6662 \\ logan@loganreed.org}  % Your phone number and email

\begin{document}

\nobibliography{ref}
\bibliographystyle{unsrt}
%----------------------------------------------------------------------------------------
%	OBJECTIVE
%----------------------------------------------------------------------------------------

\begin{rSection}{Research Interests}
Algebra, Computational Math, Numerical Analysis, Optimization, Representation Theory, Scientific Computing.


\end{rSection}
%----------------------------------------------------------------------------------------
%	EDUCATION SECTION
%----------------------------------------------------------------------------------------

\begin{rSection}{Education}
{\bf Bachelor of Science in Applied Mathematics at Texas State University} \hfill {2018-2021}
\\ 
Minor in Computer Science.
\\
3.52 GPA.
\\
\\
{\bf Master of Science in Mathematics at Rutgers University-Camden} \hfill {2021-2023}
\\
\textit{The Pólya-Sz\"ego Conjecture on Polygons: A Numerical Approach}
\\
3.96 GPA

\end{rSection}

\begin{rSection}{Publications}
  \item \bibentry{tb}
\end{rSection}
%----------------------------------------------------------------------------------------
%	TECHNICAL STRENGTHS SECTION
%----------------------------------------------------------------------------------------


% \begin{rSection}{Relevant Courses}

% \begin{tabular}{ @{} >{\bfseries}l @{\hspace{6ex}} l }
% Honors Graph Theory. \\
% Independent Study in Lie Algebra. \\
% Discrete Mathematics. \\
% Formal Language Theory. \\
% Numerical Analysis. \\ 
% Real and Complex Analysis. \\
% \end{tabular}

% \end{rSection}

%----------------------------------------------------------------------------------------
%	OBJECTIVE
%----------------------------------------------------------------------------------------


%----------------------------------------------------------------------------------------
%	WORK EXPERIENCE SECTION
%----------------------------------------------------------------------------------------

\begin{rSection}{Work History}

  \begin{rSubsection}{Private Math Tutor}{2018-}{}{}
\item I have tutored over 125 students, with 103 five star reviews.

 
\end{rSubsection} 


%------------------------------------------------

\begin{rSubsection}{Math Tutor at Math CATS}{2018-2020}{}{}
\item I tutored through the Department of Mathematics at Texas State. 
\item I gave talks at the beginning of each semester to new students with the goal of student outreach.

\end{rSubsection}

%--------------------------------------------------

\begin{rSubsection}{Math Tutor at the Math and Stats Lab at Rutgers-Camden}{2021-2022}{}{}
\item I tutored through the Department of Mathematics at Rutgers-Camden.

\end{rSubsection}

%--------------------------------------------------

\begin{rSubsection}{Part Time Lecturer}{Fall 2021}{}{}
\item I taught an Intro to College Algebra course at Rutgers-Camden. I designed the structure and material of the class independently.
\end{rSubsection}

\begin{rSubsection}{Calculus One TA}{Spring 2022}{}{}
\item I am a Teacher's Assistant for two sections of Calculus One at Rutgers-Camden. I set up the canvas pages, grade homework and exams, and maintain office hours for the students.
\end{rSubsection}

\begin{rSubsection}{Part Time Lecturer}{Fall 2022}{}{}
\item I taught an Intro to College Algebra course at Rutgers-Camden. I designed the structure and material of the class independently, except for the standardized final exam.
\end{rSubsection}

\begin{rSubsection}{Calculus Three TA}{Spring 2023}{}{}
\item I was a Teacher's Assistant for one section of Calculus Three at Rutgers-Camden. I ran the lab, graded homework and exams, and maintained office hours for the students.
\end{rSubsection}

\begin{rSubsection}{Linear Algebra TA}{Spring 2023}{}{}
\item I was a Teacher's Assistant for one section of Linear Algebra at Rutgers-Camden. I monitored canvas assignments, graded homework and exams, and maintained office hours for the students.
\end{rSubsection}

\begin{rSubsection}{Lecturer}{Fall 2023}{}{}
\item I taught a Mathematics for Liberal Arts course at Rutgers-Camden. I designed the structure and material of the class independently.
\end{rSubsection}

\begin{rSubsection}{Research Assistant}{Spring 2022-Winter 2023}{}{}
\item I assisted in the creation of software accompanying research projects.
\item I generated graphics and data to be used in the lab's research.
\end{rSubsection}


\end{rSection} 

\begin{rSection}{INDEPENDENT STUDY TOPICS}{}{} {}
 \begin{rSubsection}{K-Forcing on the Cartesian product of Simple Graphs}{}{}{}
 \item Studying the bounds on the K-Forcing number for graphs which are the Cartesian product of common families of graphs, such as paths, cycles, and trees.
 \end{rSubsection}
 
 \begin{rSubsection}{Complexes of DiGraph Homomorphisms}{}{}{}
 \item An independent study project to produce results similar to Babson and Kozlov on DiGraphs
 \end{rSubsection}
 
  \begin{rSubsection}{A Study on Minimal Prime Graphs of Simple Groups}{}{}{}
  \item An independent study project with the goal of producing new Group Theoretic results using Graph Theory
  \item The main focus was an enumeration algorithm for Triangle Free Three Colored Graphs, which correspond to Minimal Prime Graphs.
 \end{rSubsection}

 \begin{rSubsection}{Analysis}{}{}{}
 \item Working through \textit{Real and Complex Analysis} by Rudin.
 \end{rSubsection}
 
 \begin{rSubsection}{Lie Algebra}{}{}{}
  \item An independent studies course on the classical results from the algebraic field of Lie Algebra.
  \item The goal was to work through the prerequisites and eventually move to Vertex Operator Algebras.
 \end{rSubsection}
 
  \begin{rSubsection}{Vertex Operator Algebras}{}{}{}
  \item Studying from \textit{Introduction to Vertex Operator Algebras and Their Representations} by Lepowsky and Li.
 \end{rSubsection}

 \begin{rSubsection}{Algebraic Topology}{}{}{}
 \item Studying from \textit{Algebraic Topology} by tom Dieck and a book of the same name by Hatcher. 
 \end{rSubsection}

 \begin{rSubsection}{Spectral Theory}{}{}{}
 \item An independent study which resulted in studying unknown properties of the Dirichlet Laplacian.
 \item Culminated in my Master's Thesis.
 \end{rSubsection}
\end{rSection}

\begin{rSection}{EXTRA CURRICULAR ACTIVITIES}

\begin{rSubsection}{}{}{}{} 
\item Four time Dean's List recipient.
\item Head Martial Arts Instructor from 2016-2018.
\item A member of the Math Club at Texas State 2018-2020.
\item A member of the Problem Solvers Group at Texas State 2018-2019.
\item Mathematical Sciences Scholarship Award 2022.
\item Distinguished Thesis Certificate 2023.
\end{rSubsection}

\end{rSection}
%------------------------------------------------


%	EXAMPLE SECTION
%----------------------------------------------------------------------------------------

\begin{rSection}{PROGRAMMING SKILLS} \itemsep -3pt  

\item Linux, IT, MatLab, Maple, Mathematica, SQL, Git
\item 5+ years of Python/C++/C\#/JavaScript
\item 4+ years of \LaTeX

\end{rSection} 


\end{document}
